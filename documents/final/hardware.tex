
After significant development of this platform for a previous class, I came to the conclusion that it needed some significant hardware redesign to accomplish my goals of outdoor SLAM and navigation. In particular, it needed more and more accurate range measurements of its environment, and it needed a more powerful onboard processor and a more unified architecture.

To these ends, I select a Hokuyo URG (TOOD: proper model name) scanning laser range finder as the appropriate sensor. I also considered other scanning laser range finders and the Microsoft Kinect; the other lasers were more expensive, heavier, and required more power than the Hokuyo model I chose, and the Kinect was larger, heavier, used more power, and would have required a dedicated power supply to convert the robot's 8.4V battery to the 12V required by the Kinect.

The sensor selection dictated that the onboard computer needed to have host USB support, which limited my selection to a small number of embedded systems motherboards, small form-factor motherboards and the arm-based gumstix boards. Most of the small form-factor motherboards required a 12V ATX power connector to supply power, and those that didn't required a single 12V input, which would again require a separate boost convertor. The gumstix boards met most of my specifications, but have only 1 USB host port, and are difficult to expand. The embedded systems board that I found was an ALIX 3d2, which accepts a 7-20V input voltage, nearly perfectly matched to my batteries, has two USB host ports, and as a bonus has a TTL-level serial port on board, which eliminates the need for level-conversion when communicating between it and a 5V microcontroller. The ALIX is smaller than most small form-factor motherboards, and significantly larger than the gumstix, but was still of a size that was feasible to fit on the robot. I chose the ALIX and proceeded to microcontroller selection.

The requirements for my microcontroller were fairly strict; I needed at least 3 hardware UARTS, for communication with my GPS, sonars and the primary computer. An ideal microcontroller would have a 4th UART for communicating with the bluetooth serial module for external control. It also needed a significant number of digital I/O lines for reading the wheel sensors, bump sensor, and controlling the sonars, and a few analog inputs for reading the battery voltages. I considered the parallax propeller, the AtMega2560-based Arduino mega, and the Digilent Nexys II development board. The propeller meets all of the functional requirements, but from previous experience I knew that the development environment was windows-only, and diffcult for me to work with. The Arduino Mega met all of the functional requirements, and has a strong open-source tool chain that works on all major platforms. The Digilent Nexys II, being FPGA-based rather than a microtontroller, far exceeded all of the funtional requirements, but the development environment is also windows-only, and it requires significantly more development time than an equivalnet microcontroller to achieve the same results. I select the Arduino Mega because it met all of my requirements, and had a strong open-source toolchain.

I retained all of my existing sensors and interfaced them to the Arduino, including 4 wheel position sensors, 3 front sonars, 2 rear sonars, a front bump sensor, a GPS, and a compass.

I replaced the original wiring and interface circuitry, built from whatever components I had lying around, stuck into a breadboard, with properly selected componenets on a custom perfboard, with color-coded wiring.

I also replaced the original RC motor controller, which had problems switching into reverse, with a motor driver from pololu, driven with PWM generated on the arduino.

The power distribution system consists of two batteries, allowing me to isolate the motor supply from the electronics supply, while sharing a common ground. This provides enough noise isolation that the electronics are unaffected by noise from the motor.

The sensors are powered by a switching 5V regulator, while the arduino and the primary computer are connected directly to the battery supply voltage. There is also an electonic power cutoff, allowing the electonics to shut off automatically to avoid running the batteries too low. The 5V regulator has an enable line, allowing the sensors to be turned off, should the need arise.
